\section{物理基础}

\subsection{缓发中子}

在实际裂变过程中,约有$\beta$份额的裂变中子通过缓发中子先驱核衰变而来,并在瞬发中子产生后相当长一段时间才起作用,这部分中子称为缓发中子,它们对核反应堆的周期影响很大,使得核反应堆的控制成为可能。

考虑缓发中子,则中子平均寿命$\bar{l}$由瞬发中子寿命$l_0$与所有各组(通常分为6组)缓发中子寿命$(t_i + l_0)$的加权平均值,即

\begin{equation}
    \bar{l} = (1 - \beta) l_0 + \sum_{i=1}^{6} \beta_i (t_i + l_0)
\end{equation}

查${}^{235} {\rm U}$的数据,$l_0 \approx 10^{-4}\,{\rm s}$,$\displaystyle \sum_{i=1}^{6} \beta_i t_i \approx 0.1\,{\rm s}$,则$\bar{l} \approx 0.1\,{\rm s} \Rightarrow \Lambda = 0.1\,{\rm s}$。若反应性$\rho = 0.001$,则反应堆周期$T = \Lambda / \rho = 100\,{\rm s}$,即反应堆功率增加${\rm e}$倍的所需时间为$100\,{\rm s}$,完全可以使用移动控制棒的方法控制反应堆。

\subsection{反应性控制}

\begin{definition}[剩余反应性]
    堆芯中没有任何控制毒物时的反应性。
\end{definition}

\begin{definition}[后备反应性]
    冷态干净堆芯的剩余反应性。
\end{definition}

反应性控制的本质是维持堆内中子的平衡,即控制中子的产生、吸收和泄漏,具体控制手段如下。
\begin{enumerate}
    \item 中子吸收体移动控制:改变中子吸收率
    \begin{enumerate}
        \item 控制棒:控制速度快、灵活,反应性价值变化小,但对堆内中子注量率分布扰动大,对控制棒驱动机构的可靠性要求较高。
        \item 慢化剂中可溶性毒物:不对功率分布产生过大影响,但调节较慢,浓度不能过高(参照《核反应堆物理分析》第7章)。
        \item 可燃毒物棒:毒物含量岁燃耗加深而减小,第一循环结束时去掉。
    \end{enumerate}
    \item 慢化剂液位控制:改变中子产生率
    \item 燃料控制:改变中子产生率
    \item 反射层控制:改变中子泄漏率
\end{enumerate}

答题时,注意题目问的是“反应性控制的手段”(4种)还是“中子吸收体移动控制的手段”(3种)。

\subsection{核电厂稳态运行方案}

重点是冷却剂平均温度程序方案。
\begin{enumerate}
    \item 冷却剂平均温度随负荷成线性变化
    \begin{equation}
        T_{\rm av} = T_{\rm av0} + K P_{\rm H}
    \end{equation}
    \item 一回路或二回路的全部负担,由一回路和二回路共同承担
    \item 应通过向堆芯深处插入控制棒组件以补偿堆芯反应性的增加
\end{enumerate}

\subsection{习题选解}

本章所有习题均在上文给出了解答。