\section{线性控制系统的状态空间分析方法}

\subsection{状态空间表达式的建立}

\begin{align}
    & \dot{\symbfit{x}} = \symbfit{Ax} + \symbfit{B}u \\
    & y = \symbfit{Cx} + \symbfit{D}u
\end{align}

\subsubsection{由微分方程建立}

假如有

\begin{equation*}
    \dddot{y} + 6\ddot{y} + 11\dot{y} + 6y = 6u
\end{equation*}

则令

\begin{equation*}
    \begin{cases}
        x_1 = y \\ 
        x_2 = \dot{y} \\
        x_3 = \ddot{y}
    \end{cases} \Rightarrow \begin{cases}
        \dot{x}_1 = \dot{y} = x_2 \\
        \dot{x}_2 = \ddot{y} = x_3 \\
        \dot{x}_3 = -6x_1 - 11x_2 - 6x_3 + 6u
    \end{cases}
\end{equation*}

即

\begin{align*}
    \begin{bmatrix}
        \dot{x}_1 \\
        \dot{x}_2 \\
        \dot{x}_3
    \end{bmatrix} &= \begin{bmatrix}
        0 & 1 & 0 \\
        0 & 0 & 1 \\
        -6 & -11 & -6
    \end{bmatrix} \begin{bmatrix}
        x_1 \\
        x_2 \\
        x_3
    \end{bmatrix} + \begin{bmatrix}
        0 \\
        0 \\
        6
    \end{bmatrix} u \\
    y &= \begin{bmatrix}
        1 & 0 & 0
    \end{bmatrix} \begin{bmatrix}
        x_1 \\
        x_2 \\
        x_3
    \end{bmatrix}
\end{align*}

\subsubsection{由传递函数建立}

以
\begin{equation*}
    G(s) = \frac{Y(s)}{U(s)} = \frac{s^2 + 3s + 2}{s(s^2 + 7s + 12)}
\end{equation*}
为例。

\begin{enumerate}
    \item 直接转换法
    
    令
    \begin{equation*}
        \begin{cases}
            U(s) = (s^3 + 7s^2 + 12s) Q(s) \\
            Y(s) = (s^2 + 3s +2) Q(s)
        \end{cases}
    \end{equation*}

    两边同时作拉普拉斯反变换,得
    \begin{equation*}
        \begin{cases}
            u(t) = \dddot{q} + 7\ddot{q} + 12\dot{q} \\
            y(t) = \ddot{q} + 3\dot{q} + 2q
        \end{cases}
    \end{equation*}

    令
    \begin{equation*}
        \begin{cases}
            x_1 = q \\
            x_2 = \dot{q} \\
            x_3 = \ddot{q}
        \end{cases}
    \end{equation*}

    于是有
    \begin{equation*}
        \begin{cases}
            \dot{x}_1 = \dot{q} = x_2 \\
            \dot{x}_2 = \ddot{q} = x_3 \\
            \dot{x}_3 = \dddot{q} = -12x_2 - 7x_3 + u \\
            y = 2x_1 + 3x_2 + x_3
        \end{cases}
    \end{equation*}

    故
    \begin{align*}
        \begin{bmatrix}
            \dot{x}_1 \\
            \dot{x}_2 \\
            \dot{x}_3
        \end{bmatrix} &= \begin{bmatrix}
            0 & 1 & 0 \\
            0 & 0 & 1 \\
            0 & -12 & -7
        \end{bmatrix} \begin{bmatrix}
            x_1 \\
            x_2 \\
            x_3
        \end{bmatrix} + \begin{bmatrix}
            0 \\
            0 \\
            1
        \end{bmatrix} u \\
        y &= \begin{bmatrix}
            2 & 3 & 1
        \end{bmatrix} \begin{bmatrix}
            x_1 \\
            x_2 \\
            x_3
        \end{bmatrix}
    \end{align*}
    \item 部分分式法
    
    将原式分解为
    \begin{equation*}
        \frac{Y(s)}{U(s)} = \frac{1}{6}\frac{1}{s} - \frac{2}{3}\frac{1}{s+3} + \frac{3}{2}\frac{1}{s+4}
    \end{equation*}

    那么
    \begin{equation*}
        Y(s) = \frac{1}{6}\frac{1}{s}U(s) - \frac{2}{3}\frac{1}{s+3}U(s) + \frac{3}{2}\frac{1}{s+4}U(s)
    \end{equation*}

    令
    \begin{equation*}
        \begin{cases}
            X_1(s) = \frac{1}{s} U(s) \\
            X_2(s) = \frac{1}{s+3} U(s) \\
            X_3(s) = \frac{1}{s+4} U(s)
        \end{cases} \Rightarrow \begin{cases}
            sX_1(s) = U(s) \\
            sX_2(s) = -3X_2(s) + U(s) \\
            sX_3(s) = -4X_3(s) + U(s)
        \end{cases}
    \end{equation*}

    两边同时作拉普拉斯反变换,得
    \begin{equation*}
        \begin{cases}
            \dot{x}_1 = u \\
            \dot{x}_2 = -3x_2 + u \\
            \dot{x}_3 = -4x_3 + u \\
            y = \frac{1}{6}x_1 - \frac{2}{3}x_2 + \frac{3}{2}x_3
        \end{cases}
    \end{equation*}

    故
    \begin{align*}
        \begin{bmatrix}
            \dot{x}_1 \\
            \dot{x}_2 \\
            \dot{x}_3
        \end{bmatrix} &= \begin{bmatrix}
            0 & 0 & 0 \\
            0 & -3 & 0 \\
            0 & 0 & -4
        \end{bmatrix} \begin{bmatrix}
            x_1 \\
            x_2 \\
            x_3
        \end{bmatrix} + \begin{bmatrix}
            1 \\
            1 \\
            1
        \end{bmatrix} u \\
        y &= \begin{bmatrix}
            \frac{1}{6} & -\frac{2}{3} & \frac{3}{2}
        \end{bmatrix} \begin{bmatrix}
            x_1 \\
            x_2 \\
            x_3
        \end{bmatrix}
    \end{align*}

    \highlight{gray}{推荐使用部分分式法,}因为可以直接得到标准型。
    \item 传递函数与状态空间表达式之间的关系
    \begin{equation}
        G(s) = \frac{Y(s)}{U(s)} = \symbfit{C}(s\symbfit{I} - \symbfit{A})^{-1} \symbfit{B} + \symbfit{D} \label{G}
    \end{equation}
\end{enumerate}

\subsection{线性定常系统的线性变换}

经过此变换,将状态空间表达式转换为标准型,有利于判断线性定常系统的能控性和能观测性,标准型为
\begin{align}
    & \dot{\bar{\symbfit{x}}} = \bar{\symbfit{A}} \bar{\symbfit{x}} + \bar{\symbfit{B}}u \\
    & y = \bar{\symbfit{C}} \bar{\symbfit{x}} + \bar{\symbfit{D}}u
\end{align}

对于给定的非标准状态空间表达式,转换的一般步骤如下。但在此之前,最好复习一下线性代数的相关知识。

\begin{enumerate}
    \item 求特征值
    
    \begin{equation}
        \det(\lambda \symbfit{I} - \symbfit{A}) = 0 \Rightarrow \lambda_i
    \end{equation}

    并立即由此写出
    \begin{equation}
        \bar{\symbfit{A}} = \begin{bmatrix}
            \lambda_1 & & & \\
            & \lambda_2 & & \\
            & & \ddots &    \\
            & & & \lambda_n \\
        \end{bmatrix}
    \end{equation}
    \item 求特征向量
    
    \begin{equation}
        (\lambda_1 \symbfit{I} - \symbfit{A}) \symbfit{v}_i = 0 \Rightarrow \symbfit{v}_i
    \end{equation}

    线性变换矩阵由特征向量组成
    \begin{equation}
        \symbfit{P} = \begin{bmatrix}
            \symbfit{v}_1 & \symbfit{v}_2 & \cdots & \symbfit{v}_n
        \end{bmatrix}
    \end{equation}

    虽然可能大部分情况下,该矩阵均可逆,但最好计算一下$\det(\symbfit{P})$的值,说明“$\det(\symbfit{P}) \neq 0$,所以$\symbfit{P}$可逆”以求严谨,而且本来下一步也需要用到$\det(\symbfit{P})$的值。
    \item 求线性变换矩阵的逆矩阵
    
    \begin{equation}
        \symbfit{P}^{-1} = \frac{\symbfit{P}^*}{\det(\symbfit{P})}
    \end{equation}
    \item 求标准型系数矩阵
    
    \begin{align}
        & \bar{\symbfit{B}} = \symbfit{P}^{-1} \symbfit{B} \\
        & \bar{\symbfit{B}} = \symbfit{C} \symbfit{P} \\
        & \bar{\symbfit{D}} = \symbfit{D}
    \end{align}
\end{enumerate}

\subsection{线性定常系统的状态方程求解}

求解过程比较麻烦,结合例3-5掌握,需要细心。

\begin{align}
    & {\rm{e}}^{\symbfit{A}t} = \mathcal{L}^{-1} \left[(s \symbfit{I} - \symbfit{A})^{-1}\right] \\
    & x(t) = {\rm{e}}^{\symbfit{A}t} x(0) \\
    & x(t) = {\rm{e}}^{\symbfit{A}t} x(0) + \int_{0}^{t} {\rm{e}}^{\symbfit{A}(t-\tau)} \symbfit{B} \symbfit{u}(\tau) {\rm d} \tau
\end{align}

\subsection{能控性和能观测性}

对于给定的线性定常系统,有

\begin{enumerate}
    \item 状态完全能控 $\Longleftrightarrow$ $\bar{\symbfit{B}}$中无全零行
    \item 状态完全能观测 $\Longleftrightarrow$ $\bar{\symbfit{C}}$中无全零列
\end{enumerate}

\subsection{习题选解}

\begin{exercise} % 3.1
    由图可得
    \begin{align*}
        &u_i(t) = i(t)R + u_L(t) + u_C(t) \\
        &u_L(t) = L \dv{i(t)}{t} \\
        &i(t) = C \dv{u_C(t)}{t} \\
        &u_o(t) = u_C(t)
    \end{align*}
    整理可得
    \begin{align*}
        &\dv{u_C(t)}{t} = \frac{1}{C} i(t) \\
        &\dv{i(t)}{t} = -\frac{1}{L} u_C(t) - \frac{R}{L} i(t) + \frac{1}{L} u_i(t)
        &u_o(t) = u_C(t)
    \end{align*}
    写成状态空间表达式
    \begin{align*}
        \begin{bmatrix}
            \dot{u}_C(t) \\
            \dot{i}(t)
        \end{bmatrix} &= \begin{bmatrix}
            0 & \frac{1}{C} \\
            -\frac{1}{L} & -\frac{R}{L}
        \end{bmatrix} \begin{bmatrix}
            u_C(t) \\
            i(t)
        \end{bmatrix} + \begin{bmatrix}
            0 \\
            \frac{1}{L}
        \end{bmatrix} u_i(t) \\
        u_o(t) &= \begin{bmatrix}
            1 & 0
        \end{bmatrix} \begin{bmatrix}
            u_C(t) \\
            i(t)
        \end{bmatrix}
    \end{align*}
\end{exercise}

\begin{exercise} % 3.2
    令
    \begin{equation*}
        \begin{cases}
            x_1 = x(t) \\ 
            x_2 = \dot{x}(t) \\
            x_3 = \ddot{x}(t)
        \end{cases} \Rightarrow \begin{cases}
            \dot{x}_1 = \dot{x}(t) = x_2 \\
            \dot{x}_2 = \ddot{x}(t) = x_3 \\
            \dot{x}_3 = -\frac{d}{a}x_1 - \frac{c}{a}x_2 - \frac{b}{a}x_3 + \frac{1}{a}u
        \end{cases}
    \end{equation*}

    即
    \begin{align*}
        \begin{bmatrix}
            \dot{x}_1 \\
            \dot{x}_2 \\
            \dot{x}_3
        \end{bmatrix} &= \begin{bmatrix}
            0 & 1 & 0 \\
            0 & 0 & 1 \\
            -\frac{d}{a} & -\frac{c}{a} & -\frac{b}{a}
        \end{bmatrix} \begin{bmatrix}
            x_1 \\
            x_2 \\
            x_3
        \end{bmatrix} + \begin{bmatrix}
            0 \\
            0 \\
            \frac{1}{a}
        \end{bmatrix} u \\
        x &= \begin{bmatrix}
                1 & 0 & 0
        \end{bmatrix} \begin{bmatrix}
            x_1 \\
            x_2 \\
            x_3
        \end{bmatrix}
\end{align*}
\end{exercise}

\begin{exercise}
    已知
    \begin{equation*}
        G(s) = \frac{Y(s)}{U(s)} = \symbfit{C}(s\symbfit{I} - \symbfit{A})^{-1} \symbfit{B} + \symbfit{D}
    \end{equation*}
    则
    \begin{align*}
        &s\symbfit{I} - \symbfit{A} = \begin{bmatrix}
            s & -1 \\
            3 & s+4
        \end{bmatrix} \\
        &(s\symbfit{I} - \symbfit{A})^{-1} = \frac{\begin{bmatrix}
            s+4 & 1 \\
            -3 & s
        \end{bmatrix}}{s(s+4) + 3} \\
        &G(s) = \begin{bmatrix}
            10 & 0
        \end{bmatrix} \frac{\begin{bmatrix}
            s+4 & 1 \\
            -3 & s
        \end{bmatrix}}{s(s+4) + 3} \begin{bmatrix}
            0 \\
            1
        \end{bmatrix} = \frac{10}{s(s+4) + 3}
    \end{align*}
\end{exercise}

\begin{exercise} % 3.4
    \begin{equation*}
        G(s) = \frac{Y(s)}{U(s)} = \frac{2(s+3)}{(s+1)(s+2)} = \frac{4}{s+1} - \frac{2}{s+2}
    \end{equation*}
    则
    \begin{equation*}
        Y(s) = \frac{4}{s+1}U(s) - \frac{2}{s+2}U(s)
    \end{equation*}
    令
    \begin{equation*}
        \begin{cases}
            X_1(s) = \frac{1}{s} U(s) \\
            X_2(s) = \frac{1}{s+2} U(s) \\
            Y(s) = 4X_1(s) - 2X_2(s)
        \end{cases} \Rightarrow \begin{cases}
            sX_1(s) = -X_1(s) + U(s) \\
            sX_2(s) = -2X_2(s) + U(s) \\
            Y(s) = 4X_1(s) - 2X_2(s)
        \end{cases}
    \end{equation*}
    两边同时作拉普拉斯反变换,得
    \begin{equation*}
        \begin{cases}
            \dot{x}_1 = -x_1 + u \\
            \dot{x}_2 = -2x_2 + u \\
            y = 4x_1 - 2x_2
        \end{cases}
    \end{equation*}
    故
    \begin{align*}
        \begin{bmatrix}
            \dot{x}_1 \\
            \dot{x}_2
        \end{bmatrix} &= \begin{bmatrix}
            -1 & 0 \\
            0 & -2 
        \end{bmatrix} \begin{bmatrix}
            x_1 \\
            x_2 
        \end{bmatrix} + \begin{bmatrix}
            1 \\
            1 
        \end{bmatrix} u \\
        y &= \begin{bmatrix}
            4 & -2 
        \end{bmatrix} \begin{bmatrix}
            x_1 \\
            x_2 
        \end{bmatrix}
    \end{align*}
\end{exercise}

\begin{exercise} % 3.5
    令
    \begin{equation*}
        \begin{cases}
            U(s) = (s^4 - 7s^3 + 18s^2 - 20s + 8) Q(s) \\
            Y(s) = (2s^2 + 5s + 1) Q(s)
        \end{cases}
    \end{equation*}
    两边同时作拉普拉斯反变换,得
    \begin{equation*}
        \begin{cases}
            u(t) = \ddddot{q} - 7\dddot{q} + 18\ddot{q} - 20\dot{q} + 8q \\
            y(t) = 2\ddot{q} + 5\dot{q} + q
        \end{cases}
    \end{equation*}
    令
    \begin{equation*}
        \begin{cases}
            x_1 = q \\
            x_2 = \dot{q} \\
            x_3 = \ddot{q} \\
            x_4 = \dddot{q}
        \end{cases}
    \end{equation*}
    于是有
    \begin{equation*}
        \begin{cases}
            \dot{x}_1 = \dot{q} = x_2 \\
            \dot{x}_2 = \ddot{q} = x_3 \\
            \dot{x}_3 = \dddot{q} = x_4 \\
            \dot{x}_4 = \ddddot{q} = -8x_1 + 20x_2 - 18x_3 + 7x_4 + u \\
            y = x_1 + 5x_2 + 2x_3 + 0x_4
        \end{cases}
    \end{equation*}
    故
    \begin{align*}
        \begin{bmatrix}
            \dot{x}_1 \\
            \dot{x}_2 \\
            \dot{x}_3 \\
            \dot{x}_4
        \end{bmatrix} &= \begin{bmatrix}
            0 & 1 & 0 & 0 \\
            0 & 0 & 1 & 0 \\
            0 & 0 & 0 & 1 \\
            -8 & 20 & -18 & 7
        \end{bmatrix} \begin{bmatrix}
            x_1 \\
            x_2 \\
            x_3 \\
            x_4
        \end{bmatrix} + \begin{bmatrix}
            0 \\
            0 \\
            0 \\
            1
        \end{bmatrix} u \\
        y &= \begin{bmatrix}
            1 & 5 & 2 & 0
        \end{bmatrix} \begin{bmatrix}
            x_1 \\
            x_2 \\
            x_3 \\
            x_4
        \end{bmatrix}
    \end{align*}
\end{exercise}

\setcounter{exercise}{6}

\begin{exercise} % 3.7
    \begin{enumerate}
        \item 求特征值
        \begin{equation*}
            \det(\lambda \symbfit{I} - \symbfit{A}) = \begin{vmatrix}
                \lambda & -1 & 0 \\
                0 & \lambda & -1 \\
                6 & 11 & \lambda+6 
            \end{vmatrix} = \lambda^3 + 6\lambda^2 + 11\lambda + 6 = 0
        \end{equation*}
        解得
        \begin{equation*}
            \lambda_1 = -1,\,\lambda_2 = -2,\,\lambda_3 = -3
        \end{equation*}
        立即推
        \begin{equation*}
            \bar{\symbfit{A}} = \begin{bmatrix}
                -1 & 0 & 0 \\
                0 & -2 & 0 \\
                0 & 0 & -3
            \end{bmatrix}
        \end{equation*}
        \item 求特征向量
        \begin{equation*}
            (\lambda_1 \symbfit{I} - \symbfit{A}) \symbfit{v}_1 = \begin{bmatrix}
                -1 & -1 & 0 \\
                0 & -1 & -1 \\
                6 & 11 & 5
            \end{bmatrix} \begin{bmatrix}
                v_{11} \\
                v_{12} \\
                v_{13}
            \end{bmatrix} \Rightarrow \begin{cases}
                -v_{11} - v_{12} = 0 \\
                -v_{12} - v_{13} = 0 \\
                6v_{11} + 11v_{12} + 5v_{13} = 0
            \end{cases}
        \end{equation*}
        解得
        \begin{equation*}
            \symbfit{v}_1 = \begin{bmatrix}
                1 \\
                -1 \\
                1
            \end{bmatrix}
        \end{equation*}
        同理,
        \begin{equation*}
            \symbfit{v}_2 = \begin{bmatrix}
                1 \\
                -2 \\
                4
            \end{bmatrix},\,\symbfit{v}_3 = \begin{bmatrix}
                1 \\
                -3 \\
                9
            \end{bmatrix}
        \end{equation*}
        于是有
        \begin{equation*}
            \symbfit{P} = \begin{bmatrix}
                1  & 1  & 1 \\
                -1 & -2 & -3 \\
                1  & 4  & 9 \\
            \end{bmatrix}
        \end{equation*}
        且$\det(\symbfit{P}) = -2 \neq 0$,该矩阵可逆。
        \item 求线性变换矩阵的逆矩阵
        \begin{equation*}
            \symbfit{P}^{-1} = \frac{\symbfit{P}^*}{\det(\symbfit{P})} = \begin{bmatrix}
                3 & 2.5 & 0.5 \\
                -3 & -4 & -1 \\
                1 & 1.5 & 0.5 \\
            \end{bmatrix}
        \end{equation*}
        \item 求标准型系数矩阵
        \begin{align*}
            &\bar{\symbfit{B}} = \symbfit{P}^{-1}\symbfit{B} = \begin{bmatrix}
                0.5 \\
                -1 \\
                0.5
            \end{bmatrix} \\
            &\bar{\symbfit{C}} = \symbfit{C}\symbfit{P} = \begin{bmatrix}
                -1 & -3 & -5
            \end{bmatrix} \\
            &\bar{\symbfit{D}} = \symbfit{D} = 0
        \end{align*}
        故标准型为
        \begin{align*}
            &\dot{\bar{\symbfit{x}}} = \begin{bmatrix}
                -1 & 0 & 0 \\
                0 & -2 & 0 \\
                0 & 0 & -3
            \end{bmatrix} \bar{\symbfit{x}} + \begin{bmatrix}
                0.5 \\
                -1 \\
                0.5
            \end{bmatrix} u \\
            y &= \begin{bmatrix}
                -1 & -3 & -5
            \end{bmatrix} \bar{\symbfit{x}}
        \end{align*}
    \end{enumerate}
\end{exercise}

\begin{exercise} % 3.8
    \begin{enumerate}
        \item 用题3.6的方程,即
        \begin{equation*}
            \begin{bmatrix}
                \dot{x}_1 \\
                \dot{x}_2
            \end{bmatrix} = \begin{bmatrix}
                0 & 1 \\
                -1 & -2 
            \end{bmatrix} \begin{bmatrix}
                x_1 \\
                x_2 
            \end{bmatrix} + \begin{bmatrix}
                0 \\
                1 
            \end{bmatrix} u
        \end{equation*}
        且已知$x_1(0)$,$x_2(0)$和$u(\tau) = 1$,求系统的解。
        \begin{align*}
            &s\symbfit{I} - \symbfit{A} = \begin{bmatrix}
                s & -1 \\
                1 & s+2
            \end{bmatrix} \\
            &(s\symbfit{I} - \symbfit{A})^{-1} = \frac{\begin{bmatrix}
                s+2 & 1 \\
                -1 & s
            \end{bmatrix}}{(s+1)^2} \\
            &{\rm e}^{\symbfit{A}t} = \mathcal{L}^{-1}\left[(s\symbfit{I} - \symbfit{A})^{-1}\right] = \begin{bmatrix}
                {\rm e}^{-t}+t{\rm e}^-t & t{\rm e}^-t \\
                -t{\rm e}^-t & {\rm e}^{-t}-t{\rm e}^-t
            \end{bmatrix} \\
            &\int_{0}^{t} {\rm e}^{\symbfit{A}(t-\tau)} \begin{bmatrix}
                0 \\
                1
            \end{bmatrix} u(\tau) {\rm d}\tau = \int_{0}^{t} \begin{bmatrix}
                (t-\tau) {\rm e}^{-(t-\tau)} \\
                {\rm e}^{-(t-\tau)} - (t-\tau) {\rm e}^{-(t-\tau)}
            \end{bmatrix} {\rm d}\tau = \begin{bmatrix}
                1-(t+1){\rm e}^{-t} \\
                t{\rm e^{-t}}
            \end{bmatrix}
        \end{align*}
        故,系统的解为
        \begin{equation*}
            \begin{bmatrix}
                x_1 \\
                x_2
            \end{bmatrix} = {\rm e}^{\symbfit{A}t} \begin{bmatrix}
                x_1(0) \\
                x_2(0)
            \end{bmatrix} + \int_{0}^{t} {\rm e}^{\symbfit{A}(t-\tau)} \begin{bmatrix}
                0 \\
                1
            \end{bmatrix} u(\tau) {\rm d}\tau = \begin{bmatrix}
                ({\rm e}^{-t}+t{\rm e}^{-t})x_1(0) + t{\rm e}^{-t}x_2(0) + 1 - (t+1){\rm e}^{-t} \\
                -t{\rm e}^{-t}x_1(0) + ({\rm e}^{-t} - t{\rm e}^{-t})x_2(0) + t{\rm e}^{-t}
            \end{bmatrix}
        \end{equation*}
        \item 用题3.8的方程,即
        \begin{equation*}
            \begin{bmatrix}
                \dot{x}_1 \\
                \dot{x}_2 \\
                \dot{x}_3
            \end{bmatrix} = \begin{bmatrix}
                2 & 1 & 0 \\
                0 & 2 & 1 \\
                0 & 0 & 2
            \end{bmatrix} \begin{bmatrix}
                x_1 \\
                x_2 \\
                x_3
            \end{bmatrix}
        \end{equation*}
        且已知$x_1(0)$,$x_2(0)$和$x_3(0)$,求系统的解。
        \begin{align*}
            &s\symbfit{I} - \symbfit{A} = \begin{bmatrix}
                s-2 & -1 & 0 \\
                0 & s-2 & -1 \\
                0 & 0 & s-2
            \end{bmatrix} \\
            &(s\symbfit{I} - \symbfit{A})^{-1} = \begin{bmatrix}
                \frac{1}{s-2} & \frac{1}{(s-2)^2} & \frac{1}{(s-2)^3} \\
                0 & \frac{1}{s-2} & \frac{1}{(s-2)^2} \\
                0 & 0 & \frac{1}{s-2}
            \end{bmatrix} \\
            &{\rm e}^{\symbfit{A}t} = \mathcal{L}^{-1}\left[(s\symbfit{I} - \symbfit{A})^{-1}\right] = {\rm e}^{2t} \begin{bmatrix}
                1 & t & \frac{t^2}{2} \\
                0 & 1 & t \\
                0 & 0 & 1
            \end{bmatrix}
        \end{align*}
        故,系统的解为
        \begin{equation*}
            \begin{bmatrix}
                x_1 \\
                x_2 \\
                x_3
            \end{bmatrix} = {\rm e}^{\symbfit{A}t} \begin{bmatrix}
                x_1(0) \\
                x_2(0) \\
                x_3(0)
            \end{bmatrix} = {\rm e}^{2t} \begin{bmatrix*}[r]
                x_1(0) + tx_2(0) + t^2 x_3(0) \\
                x_2(0) + tx_3(0) \\
                x_3(0)
            \end{bmatrix*}
        \end{equation*}
    \end{enumerate}
\end{exercise}

\begin{exercise} % 3.9
    \begin{enumerate}
        \item $\bar{\symbfit{B}}$中有全零行,系统不完全能控;
        \item $\bar{\symbfit{C}}$中有全零列,系统不完全能观测。
    \end{enumerate}
\end{exercise}
